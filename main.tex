\documentclass[a4paper, 14pt]{article}
\usepackage{fontspec, extsizes, geometry, setspace, titlesec, fancyhdr, graphicx, float, setspace, caption, array, tabularx, ulem, indentfirst, ragged2e}
\usepackage[ukrainian]{babel} % підтримка мов
\usepackage[dotinlabels]{titletoc} 
\setmainfont[Ligatures=TeX]{Times New Roman}
\geometry{a4paper,total={165mm,257mm},left=3cm,top=2cm,bottom=2cm,right=1.5cm} % cтавить береги та формат а4
\def\changemargin#1#2{\list{}{\rightmargin#2\leftmargin#1}\item[]}\let\endchangemargin=\endlist % зручна команда для виставлення відступів
\makeatletter\newcommand\Dotfill{\leavevmode\leaders\hb@xt@0.25em{\hss.\hss}\hfill}\makeatother % команда для ставлення точок
\let\stdsection\section\renewcommand\section{\newpage\stdsection} % Новая секція -> нова сторінка
\addto\captionsenglish{\renewcommand{\figurename}{Рис.}} % підпис картинок
\titleformat{\section}[display]{\filcenter}{\bfseries{РОЗДІЛ \thesection}}{0pt}{\bfseries\MakeUppercase} % Изменение заголовка всех разделов
\renewcommand{\thesubsection}{\arabic{section}.\arabic{subsection}} % Чиним номер подраздела
\titleformat{\subsection}{\filcenter}{\bfseries \thesubsection. }{0pt}{\bfseries}{} %чиним название подраздела 
\captionsetup{labelsep=period} % "Рис. 1." вместо "Рис. 1:"
\fancyhf{}\renewcommand{\headrulewidth}{0pt}\newcommand{\changefont}{\fontsize{14}{14}\selectfont}\fancyhead[R]{\changefont \thepage}\fancypagestyle{plain}{\fancyhf{}\fancyhead[R]{\changefont \thepage}\renewcommand{\headrulewidth}{0pt}\renewcommand{\footrulewidth}{0pt}}\pagestyle{fancy} %номер страницы справа сверху на всех страничках [это ужас]
\linespread{1.43} % Інтервал абзацу полуторний
\renewcommand{\contentsname}{ЗМІСТ} %изменяем название странички с содержанием
\def\numberline#1{#1. } % Фикс чтобы названия не налезали друг на друга в содержании
\titlecontents{section}[0pt]{\normalfont}{РОЗДІЛ \thecontentslabel. }{}{\Dotfill \contentspage} % оформление разделов, точек в содержании
\titlecontents{subsection}[15pt]{\normalfont}{\thecontentslabel. }{}{\Dotfill \contentspage} % оформление подразделов, точек в содержании
\titlecontents{subsubsection}[30pt]{\normalfont}{\thecontentslabel. }{}{\Dotfill \contentspage} % оформление подподразделов, точек в содержании
\counterwithin{table}{section} % нумерация таблиц с номером раздела
\counterwithin{table}{section} % нумерация таблиц с номером раздела
\usepackage[none]{hyphenat} \justifying\sloppy % щоб текст розтягувався
\setlength{\parindent}{0.5in} % виставляємо відступ абзацу

% ------------------------------------------- Преамбула закінчилась -------------------------------------------

% ------------------------------------------- Титулка ---------------------------------------------------------
\begin{document}
% Титулка
\thispagestyle{empty}
\begin{center}
Міністерство освіти і науки України\\
Харківський національний університет радіоелектроніки \par
\null\par
Кафедра програмної інженерії \par
\null\par\null\par\null\par
КУРСОВА РОБОТА\\
ПОЯСНЮВАЛЬНА ЗАПИСКА\\
з дисципліни ``Об'єктно-орієнтоване програмування''\\
<НАЗВА> 
\end{center}
\par\null\par\null
\begin{changemargin}{-0.25cm}{0cm}
\begin{tabular}{ p{20em} p{11em} } 
Керівник ,  <посада> & <прізвище, ініціали > \\
Студент гр. <шифр групи> & <прізвище, ініціали > \\
\end{tabular}
\par\null\par\null
\begin{tabular}{ l } 
Комісія: \\
\end{tabular}
\end{changemargin}
\begin{changemargin}{1cm}{0cm}
\begin{tabular}{ p{12em} p{7em} p{8em} }
    <посада> & \underline{\makebox[7em][c]{}} & <прізвище, ініціали> \\
    <посада> & \underline{\makebox[7em][c]{}} & <прізвище, ініціали> \\
    <посада> & \underline{\makebox[7em][c]{}} & <прізвище, ініціали> \\
\end{tabular}
\vspace*{\fill}\end{changemargin}
\begin{center}
Харків -- \the\year{}
\end{center}
\newpage
% ------------------------------------------- Титулка закінчилась ---------------------------------------------

% ------------------------------------------- Аркуш завдання --------------------------------------------------
\begin{center}
    ХАРКІВСЬКИЙ НАЦІОНАЛЬНИЙ УНІВЕРСИТЕТ РАДІОЕЛЕКТРОНІКИ
    \begin{changemargin}{1cm}{0cm}
        \begin{tabular}{ l l }
    Кафедра & \textit{програмної інженерії} \\
Рівень вищої освіти & \textit{перший (бакалаврський)} \\
Дисципліна & \textit{Об’єктно-орієнтоване програмування} \\
Спеціальність & 1\textit{21 Інженерія програмного забезпечення} \\
Освітня програма & \textit{Програмна інженерія} \\
\end{tabular}
    \end{changemargin}
\begin{tabularx}{\textwidth} { 
   >{\raggedright\arraybackslash}X 
   >{\centering\arraybackslash}X 
   >{\raggedleft\arraybackslash}X  }
 Курс \underline{\makebox[5em][c]{\textit{1}}}  & Група \underline{\makebox[5em][c]{\textit{ПЗПІ-23-X}}} & Семестр \underline{\makebox[5em][c]{\textit{2}}}\\
\end{tabularx}
\null\par\null
\textit{\textbf{ЗАВДАННЯ \\
на текстовий проєкт студента}} \par
\underline{\makebox[\textwidth][c]{<прізвише, ім'я, по батькові>}} \\
\scriptsize{(Прізвище, Ім'я, По батькові)} \\
\end{center}
1 Тема проєкту: \\
\uline{\makebox[\textwidth][c]{<тема проєкту>}} \\ 
2 Термін здачі студентом закінченого проекту: \textbf{\textit{``\underline{08}'' - червня - 2024 р.}} \\
3 Вихідні дані до проекту: \par
\uline{<Завдання на курсову роботу> <Завдання на курсову роботу> <Завдання на курсову роботу> <Завдання на курсову роботу> < Завдання на курсову роботу > < Завдання на курсову роботу > <Завдання на курсову роботу>}  \par \null \par \noindent
Зміст розрахунково-пояснювальної записки: \par
\uline{<Вступ, опис вимог, проектування програми, інструкція користувача,
висновки> <Вступ, опис вимог, проектування програми, інструкція,
висновки >} \\
\newpage
% ------------------------------------------- Аркуш завдання закінчився ---------------------------------------

% ------------------------------------------- Календарний план ------------------------------------------------
\noindent
\begin{center}
    КАЛЕНДАРНИЙ ПЛАН \par \null \par
    \end{center}
  \begin{tabular}{|p{1em} | p{17em} | p{11em}|}
     \hline
        \multicolumn{1}{|c|}{\textit{№}} & \multicolumn{1}{c}{\textit{Назва етапу}} & \multicolumn{1}{|c|}{\textit{Термін виконання}} \\ \hline
    1 & Видача теми, узгодження і затвердження теми & 13.02.2024 - 15.03.2024 р. \\ \hline
    2 & Формулювання вимог до програми & \hspace{1em}.\hspace{1em}.2024 – \hspace{1em}.\hspace{1em}.2024 р. \\ \hline
    3 & Розробка підсистеми зберігання та пошуку даних. & \hspace{1em}.\hspace{1em}.2024 – \hspace{1em}.\hspace{1em}.2024 р. \\ \hline
    4 & Розробка функцій \ldots & \hspace{1em}.\hspace{1em}.2024 – \hspace{1em}.\hspace{1em}.2024 р. \\ \hline
    5 & Розробка функцій зберігання та завантаження даних & \hspace{1em}.\hspace{1em}.2024 – \hspace{1em}.\hspace{1em}.2024 р. \\ \hline
    6 & Тестування і доопрацювання розробленої програмної системи. & \hspace{1em}.\hspace{1em}.2024 – \hspace{1em}.\hspace{1em}.2024 р. \\ \hline
    7 & Оформлення пояснювальної записки, додатків, графічного матеріалу & \hspace{1em}.\hspace{1em}.2024 – \hspace{1em}.\hspace{1em}.2024 р. \\ \hline
    8 & Захист & \hspace{1em}.\hspace{1em}.2024 – \hspace{1em}.\hspace{1em}.2024 р. \\ \hline
  \end{tabular}

\par \null \par \null \par \null \par \null \par \null \par \noindent
\begin{tabularx}{\textwidth} { 
   >{\raggedright\arraybackslash}X 
   >{\raggedleft\arraybackslash}X  }
    Cтудент \underline{\hspace{10em}} \\
    \\
    Керівник \underline{\hspace{10em}} & \underline{\hspace{10em}} \\
     & \scriptsize{(Прізвище, Ім'я, По батькові)} \\
     << 21 >> \underline{\makebox[5em][l]{ лютого}} 2024 р.
\end{tabularx}
% ------------------------------------------- Календарний план закінчився -------------------------------------

% ------------------------------------------- Реферат ---------------------------------------------------------
\section*{РЕФЕРАТ}
Пояснювальна записка до курсової роботи: 44 с., 10 рис., 3 табл., 2
додатки, 7 джерел. \par
ПОКУПЕЦЬ, МАГАЗИН, ЗВІТ, ООП, .NET, МОВА C\# \par
Метою роботи є розробка програми «Довідник покупця», яка буде
надавати користувачу довідки про товари та магазини. \par
В результаті отримана програма, що дозволяє зберігати список магазинів,
характеристики кожного магазину, такі як: назва, спеціалізація, адреса,
телефон, час роботи, вид власності. Є можливість утворювати нові списки
магазинів, додавати, видаляти та редагувати магазини. Також, є функція
формування звіту у Microsoft Excel. \par
В процесі розробки використано середовище розробки Microsoft Visual
Studio 2022, фреймворк Windows Forms, платформа .NET 8.0, мова
програмування C\#.
% ------------------------------------------- Реферат закінчився ----------------------------------------------

% ------------------------------------------- Зміст -----------------------------------------------------------
\tableofcontents %генерація змісту
% ------------------------------------------- Зміст завершився ------------------------------------------------

% ------------------------------------------- Вступ -----------------------------------------------------------
\section*{\textbf{ВСТУП}}
\addcontentsline{toc}{section}{ВСТУП} %додаємо сторінку Вступу до змісту
Ваш ВСТУП
% ------------------------------------------- Вступ закінчився ------------------------------------------------

% ------------------------------------------- Ну і далі сама ваша курсова -------------------------------------
\section{Основна частина}
Одна з провідних глобальних компаній надає широкий спектр послуг, а саме закупівлю-продаж, пасажирські перевезення та кабельне телебачення.

Хочемо підкорювати стильних особистостей комфортом глобального громадянства і прагнемо розвивати передачу даних, послуги доставки кореспонденції і юридичний захист ваших прав разом із замовниками. У своїй діяльності товариство застосовує універсальні сучасні технології новаторства, відкриттів та безпрограшних домовленостей. Незмінно зміцнює позиції широкий вибір інновацій: кімнатні й садові рослини, кредити для малого та середнього бізнесу і мобільний голосовий зв'язок для гостей столиці і домашніх улюбленців. Систематичне вдосконалення, фінансово відповідальні ціни, модернізація топ-менеджерів та синергетичне поєднання забезпечили організації успіх і провідну роль на світовому ринку.

\begin{figure}[h]
    \centering
    \includegraphics[width=0.25\textwidth]{example-image-a}
    \caption{Правильно подпись делает, между прочим!}
    \label{fig:mesh1}
\end{figure} 
\begin{figure}[h]
    \centering
    \includegraphics[width=0.25\textwidth]{example-image-a}
    \caption{Правильно подпись делает, между прочим!}
    \label{fig:1}
\end{figure} 
\label{sec:main}
\subsection{Подраздел}
\label{subsec:main}
\subsubsection{Подподраздел}
\label{subsubsec:main}
\begin{figure}[h]
    \centering
    \includegraphics[width=0.25\textwidth]{example-image-a}
    \caption{Правильно подпись делает, между прочим!}
    \label{fig:mesh1}
\end{figure} 
\section{очень-очень-очень-очень-очень-очень-очень-очень длинное название раздела (такое бывает)}
Как было сказано в разделе \ref{sec:main}...\\
Как было сказано в подразделе \ref{subsec:main}...\\
Как было сказано в подподразделе \ref{subsubsec:main}...\\
% ------------------------------------------- Курсова завершилась ---------------------------------------------

% ------------------------------------------- Перелік джерел посилання ----------------------------------------
\addcontentsline{toc}{section}{ПЕРЕЛІК ДЖЕРЕЛ ПОСИЛАННЯ} 
\section*{ПЕРЕЛІК ДЖЕРЕЛ ПОСИЛАННЯ}
\begin{enumerate}
    \item Дичківська О. О. Інноваційний менеджмент : конспект лекцій. Київ :
ДІА, 2018. 82 с.
    \item Мороз І. С., Василенко Н. Ю. Маркетинг : конспект лекцій. Київ :
Молодь, 2016. 102 с.
\end{enumerate}
% ------------------------------------------- Перелік джерел посилання закінчився -----------------------------

\end{document}

